\section{Thesis Outline}
% We organize the remaining of this thesis in the following manner. In the first three chapters, we describe our research in the area of “visual perception in data communication”. In Chapter 2, we research how statistics and graphics play a role in
% 9 opinion change [123 -- JK citation to original paper]; then, in Chapter 3, we discuss our work to empirically evaluate
% how deceptive visualizations can lead to misinterpretation of the message [126]. Next, in Chapter 4, we discuss whether or not more expressive visualizations (as compared to the standard ones) elicit more empathy among its readers [33].
% This is followed by our work around the second sub-theme, i.e.,, “visual per- ception in data analysis”. In Chapter 5, we present our study on how humans perceive similarity between large sets of scatterplots and how that compares with computational measures [124]. Next, in Chapter 6, we extend this work by building an open-source framework called StatScan that can help future researchers quickly setup data analysis pipeline to compare statistical metrics with each other, or with human perceptual response.
% Furthermore, in Chapter 7, we present a summary of all our works while discussing their key findings, implications and open issues, that can lead to potential research directions for future researchers. Lastly, in Chapter 8, we discuss the conclusions and future works.
\infuse (\chapref{chap:infuse}), \prospector (\chapref{chap:prospector}), and the model diagnostic workflow using instance-level explanations (\chapref{chap:explainer}).
After that the progress on the current stage is described in \chapref{chap:summary} and \todo{TODO}.
\todo{outline -- summary of main contributions at the beginning of each chapter (Research Question: and Key Findings:)}