\section{Use Case: Machine Learning for Health Care}
In this thesis we will mostly explore use cases from the medical domain\footnote{However, results are easily transferable to other domains as well.}.
It is quite interesting to see parallels between the decision making process of doctors and machine learning.

Over thousands of years the health care domain has evolved into what we now know as evidence based medicine.
That is, new drugs can only be approved after long studies that show significant improvements and methods have been developed and refined over time to improve their measurable effectiveness.

For example, when a patient is examined by a doctor, the doctor follows a structured approach similar to a decision tree.
The first few nodes in the tree set up the context: age, ethnicity, weight, visual appearance, \etc.
A young patient has a vastly different set of potential health risks than an old patient.
Likewise, a physically active patient has different health risks than an obese patient.
Only after those initial features come symptoms into play.
This pattern is also reflected in the documentation of a patient encounter, the doctor's note.
Doctor's notes are structured in a way that reinforces this decision tree approach by putting the context establishing features first.

However, when it comes \todo{}

* Doctor decision tree (doctor note, limitations, residents vs. attending intuition; seniors might not be able to fully explain decisions made because of learned examples)
* inconsistent coding NDC, IDC, hierarchies
