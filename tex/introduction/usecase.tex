\section{Use Case: Machine Learning for Health Care}
In this thesis we will mostly explore use cases from the medical domain\footnote{However, results are easily transferable to other domains as well.}.
It is quite interesting to see parallels between the decision making process of doctors and machine learning.

Over thousands of years the health care domain has evolved into what we now know as evidence based medicine.
That is, new drugs can only be approved after long studies that show significant improvements and methods have been developed and refined over time to improve their measurable effectiveness.

For example, when a patient is examined by a doctor, the doctor follows a structured approach similar to a decision tree.
The first few nodes in the tree set up the context: age, ethnicity, weight, visual appearance, \etc.
A young patient has a vastly different set of potential health risks than an old patient.
Likewise, a physically active patient has different health risks than an obese patient.
Only after those initial features come symptoms into play.
This pattern is also reflected in the documentation of a patient encounter, the doctor's note.
Doctor's notes are structured in a way that reinforces this decision tree approach by putting the context establishing features first.

When it comes to symptoms, doctors often have to weigh different plausible diagnoses against each other, as the same symptoms can appear from different causes.
This is called \emph{differential diagnosis}.
In order to determine which diagnosis has the highest likelihood, Bayesian reasoning is applied~\cite{mdbook}.
That is, the conditional probabilities of symptoms with respect to different diagnoses are compared to each other.
The probabilities are determined through historical records.
However, the resulting rules are often simplified to be memorizable by humans.

While residents (recent medical doctorate graduates) follow those rules religiously, attending physicians (more senior medical doctors) might sometimes follow their intuition, acquired through years of patient interactions.
Thus, such a medical decision might not be fully explainable through empirical statistics alone anymore.
This is equivalent to, for example, a Deep Neural Network that learned some transformations through its hidden layers that lead to hard to interpret interactions of the initial input features.
