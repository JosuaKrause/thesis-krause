%%
%% Place here your \usepackage's. Some recommended packages are already included.
%%

% Graphics:
\usepackage[final]{graphicx}
%\usepackage{graphicx} % use this line instead of the above to suppress graphics in draft copies
%\usepackage{graphpap} % \defines the \graphpaper command

% Indent first line of each section:
\usepackage{indentfirst}

% Good AMS stuff:
\usepackage{amsthm} % facilities for theorem-like environments
\usepackage[tbtags]{amsmath} % a lot of good stuff!

% Fonts and symbols:
\usepackage{amsfonts}
\usepackage{amssymb}

\usepackage[export]{adjustbox}
\usepackage{xspace}
\usepackage{algorithmic}
\usepackage{algorithm}
\usepackage{microtype}
% \usepackage{subfigure}
\usepackage{color}
% \usepackage{todonotes}
\usepackage{url}
\usepackage{siunitx}
\usepackage{tcolorbox}

\newfloat{algorithm}{t}{lop}

% Formatting tools:
%\usepackage{relsize} % relative font size selection, provides commands \textsmalle, \textlarger
%\usepackage{xspace} % gentle spacing in macros, such as \newcommand{\acims}{\textsc{acim}s\xspace}

% Page formatting utility:
%\usepackage{geometry}
\usepackage{multirow}

%
\usepackage{listings}


%
\usepackage[numbers,sort]{natbib}

\usepackage[all,cmtip]{xy}

\usepackage{hyperref}
\hypersetup{
    colorlinks=true,
    linkcolor=blue,
    filecolor=blue,
    urlcolor=blue,
    citecolor=blue
}

% additional packages from proposal
% % \renewcommand{\familydefault}{\sfdefault}
% \usepackage[scaled=0.85]{beramono}
\usepackage{wrapfig}
% % \usepackage{caption}
\usepackage{subcaption}
\usepackage{tikz}
% \usepackage{natbib}
% % \usepackage[usenames,dvipsnames]{xcolor}
% % \usepackage{float}
% % \usepackage[rflt]{floatflt}
% \usepackage{hyperref}
% \hypersetup{
%     colorlinks = true,
%     allcolors = {black}
%     % citecolor = {blue},
%     % anchorcolor = {black}
% }

\usepackage{enumitem}
%%%

%%
%% Place here your \newcommand's and \renewcommand's. Some examples already included.
%%
%\newcommand{\acims}{\textsc{acim}s\xspace}
\newcommand{\Mspace}        {{\mathbb M}}
\newcommand{\Rspace}        {{\mathbb R}}
\newcommand{\Cspace}        {{\mathbb C}}

\newcommand{\Mo}        {{\hat M}}
\newcommand{\Ms}        {{\tilde M}}
\newcommand{\Do}          {{\hat D}}
\newcommand{\Ds}        {{\tilde D}}
\newcommand{\doo}          {{\hat d}}
\newcommand{\dss}        {{\tilde d}}
\newcommand{\w}        {{\mathbf w}}

% general
\newcommand{\reffig}[1]{{Figure~\ref{#1}}}
\newcommand{\refchap}[1]{{Chapter~\ref{#1}}}
\newcommand{\refsec}[1]{{Section~\ref{#1}}}
\newcommand{\reftab}[1]{{Table~\ref{#1}}}
\newcommand{\refapp}[1]{{Appendix~\ref{#1}}}
\newcommand{\refeq}[1]{{Equation~\ref{#1}}}
\newcommand{\refalg}[1]{{Algorithm~\ref{#1}}}
\newcommand{\myparagraph}[1]{\noindent \textbf{#1}}
\newcommand{\highlight}[1]{{\color{black}#1}}

\newenvironment{contributions}[1]{
\begin{center}
\begin{tcolorbox}[title=Summary:,width=0.8\linewidth]
\textbf{Research Question:}\\\emph{#1}

\textbf{Key Findings:}
\begin{itemize}
}{
\end{itemize}
\end{tcolorbox}
\end{center}
}

\newcommand{\todo}[1]{\textcolor{blue}{[TODO] #1}\PackageWarning{definitions}{[TODO] #1}}
\newcommand{\expand}{\textcolor{green}{[EXPAND]}\PackageWarning{definitions}{EXPAND}}

\newcommand{\eg}{\emph{e.g.}\xspace}
\newcommand{\ie}{\emph{i.e.}\xspace}
\newcommand{\etc}{\emph{etc.}\xspace}
\newcommand{\wrt}{\emph{w.r.t.}\xspace}
\newcommand{\etal}{\emph{et~al.}\xspace}

\newcommand{\figref}[2][]{Figure~\ref{#2}#1\xspace}
\newcommand{\chapref}[1]{Chapter~\ref{#1}\xspace}
\newcommand{\secref}[1]{Section~\ref{#1}\xspace}
\newcommand{\infuse}{\emph{INFUSE}\xspace}

\newcommand{\prospector}{\emph{Prospector}\xspace}

\newcommand{\tabA}{Statistical Summary View\xspace}
\newcommand{\tabB}{Explanation Explorer\xspace}
\newcommand{\tabC}{Item Level Inspector\xspace}

\DeclareMathOperator*{\argmin}{arg\,min}
\DeclareMathOperator*{\argmax}{arg\,max}

% \newcommand{\ainfo}[1]{\phantom{#1}}
\newcommand{\ainfo}[1]{#1}

%%
%% Place here your \newtheorem's:
%%

%% Some examples commented out below. Create your own or use these...
%%%%%%%%%\swapnumbers % this makes the numbers appear before the statement name.
%\theoremstyle{plain}
%\newtheorem{thm}{Theorem}[chapter]
%\newtheorem{prop}[thm]{Proposition}
%\newtheorem{lemma}[thm]{Lemma}
%\newtheorem{cor}[thm]{Corollary}

%\theoremstyle{definition}
%\newtheorem{define}{Definition}[chapter]

%\theoremstyle{remark}
%\newtheorem*{rmk*}{Remark}
%\newtheorem*{rmks*}{Remarks}

%% This defines the "proo" environment, which is the same as proof, but
%% with "Proof:" instead of "Proof.". I prefer the former.
%\newenvironment{proo}{\begin{proof}[Proof:]}{\end{proof}}
