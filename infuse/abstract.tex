\begin{quote}
\textit{Predictive modeling techniques are increasingly being used by
data scientists to understand the probability of predicted outcomes.
However, for data that is high-dimensional, a critical step in predictive
modeling is determining which features should be included in the models.
Feature selection algorithms are often used to
remove non-informative features from models.
However, there are many different classes of feature selection algorithms.
Deciding which one to use is problematic as the algorithmic output
is often not amenable to user interpretation.
This limits the ability for users to utilize their
domain expertise during the modeling process.
To improve on this limitation, we developed \infuse ,
a novel visual analytics system designed to help analysts
understand how predictive features are being ranked across
feature selection algorithms, cross-validation folds, and classifiers.
We demonstrate how our system can lead to important insights
in a case study involving clinical researchers predicting patient
outcomes from electronic medical records.
}\end{quote}

\begin{contributions}{How do different feature selection strategies compare to each other?}
\item Different strategies prefer different, equally reasonable, feature sets without having a significant impact on predictive performance.
\item Inspecting and comparing alternate settings lets machine learning experts develop insights that overwrite their initial intuitions.
\item Rankings from feature selection strategies are not informative enough to help understand model decisions.
\end{contributions}

\begin{quote}
\textit{Josua Krause, Adam Perer, Enrico Bertini}
\end{quote}